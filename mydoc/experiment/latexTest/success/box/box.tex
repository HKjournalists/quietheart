\documentclass[11pt,a4paper]{article}

%中文支持
\usepackage{fontspec}
%这里使用的是宋体字
\setmainfont{AR PL UMing CN}
%中文断行,如果没有这个中文断行就可能会有问题
\XeTeXlinebreaklocale "zh"
\XeTeXlinebreakskip = 0pt plus 1pt 

\begin{document}
%最简单的盒子是 \mbox 和 \fbox。前者把一组对象组合起来,后者在此基础上加了个边框。
%这里因为没有空行,所以mbox和fbox里面的内容连同hello都在同一行内
hello,box1:
\mbox{123 456 789 101112}
\fbox{123 456 789 101112}

%稍复杂的 \makebox 和 \framebox 提供了宽度和对齐方式控制选项。这里用 l、r、s 分别代表居左、居右和分散对齐。
%语法:[宽度][对齐方式]{内容}
hello box2:
%为什么分散的参数不好用呢????

%无框
\makebox[100pt][l]{居左}

\makebox[100pt][r]{居右}

\makebox[100pt][c]{居中}

\makebox[100pt][s]{分散}

%有框
\framebox[100pt][l]{居左}

\framebox[100pt][r]{居右}

\framebox[100pt][c]{居中}

\framebox[100pt][s]{分散}

%大一些的对象比如整个段落可以用 \parbox 命令和 \minipage 环境,两者语法类似,也提供了对齐方式和宽度的选项。但是这里的对齐方式是指与周围内容的纵向关系,用 t、c、b 分别代表居顶、居中和居底对齐。
hello box3:

%parbox没有边框
%语法:[对齐方式]{宽度}{内容}
\parbox[c]{90pt}{锦瑟无端五十弦,\\一弦一柱思华年。}李商隐
%这个可以编译通过,但是格式显示的有些问题不知道为什么
%\begin{parbox}[c]{90pt}
%锦瑟无端五十弦,\\一弦一柱思华年。
%\end{parbox}李商隐

%minipage没有边框
%这个不知道为什么编译通不过去
%\minipage[c]{90pt}{锦瑟无端五十弦,\\一弦一柱思华年。}李商隐
%应该用这个
%\begin{minipage}[position]{width}
%text
%\end{minipage}
\begin{minipage}[c]{90pt}
锦瑟无端五十弦,\\一弦一柱思华年。
\end{minipage}李商隐

\end{document}
