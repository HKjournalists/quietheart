%%%%%文档类(documentclass)有三种:article、report、book%%%%%
%\documentclass[options]{class} %文档类声明

%\usepackage[options]{package}  %引入宏包

%%%%%文档正文部分的开头通常有标题、作者、摘要等信息,之后是章%%%%
%%%%%节等层次结构,内容则散布于层次结构之间%%%%
%\begin{document}               %正文
%\title{标题}
%\author{作者}
%\today
%\maketitle

%\begin{abstract}
%...摘要部分...
%\end{abstract}

%%%%%常用的层次结构%%%%%
%\chapter{...}
%\section{...}
%\subsection*{...}%不让此章节显示在目录中加*
%\subsubsection{...}

%%article 中没有 chapter,而report 和 book 则支持上面所有层次。


%\end{document}


\documentclass[10pt,titlepage]{report} %文档类声明
\begin{document}               %正文
%LATEX 中有三种列表环境:itemize、enumerate、description,分别如下


\begin{itemize}%没有数字标号的列表
    \item C++
		\begin{enumerate}
			\item c

			This is a very good language.%内层的item的内容,这里一定要有前面的空行,否则就和前面的字符c在同一行了。
			\item c++

			Extension of the language of c.
			\item function
		\end{enumerate}
    \item Java
    \item HTML
\end{itemize}


\begin{enumerate}%带有数字标号的列表,如果没有前面的两个空行也没有看出什么不同,但是最好有。
\item C++
\item Java
\item HTML
\end{enumerate}


\begin{description}
\item{C++} a programming language
\item{Java} another programming language
\item{HTML} another programming language for network
\end{description}
\end{document}
