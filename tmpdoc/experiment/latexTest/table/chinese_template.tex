\documentclass[11pt,a4paper]{article}

%中文支持
\usepackage{fontspec}
%这里使用的是宋体字
\setmainfont{AR PL UMing CN}
%中文断行,如果没有这个中文断行就可能会有问题
\XeTeXlinebreaklocale "zh"
\XeTeXlinebreakskip = 0pt plus 1pt 

\begin{document}

%tabular环境提供了最简单的表格功能。
%\hline命令代表横线
%|l|c|r中:|代表竖线,|l|c|r|表示分别是每个栏目用左,中,右方式对齐。
%表格内容用&来分栏.


\begin{tabular}{|l|c|r|}
\hline
操作系统 & 发行版 & 编辑器 \\
\hline
Windows & MikTeX & TeXnicCenter \\
\hline
Unix/Linux & TeX Live & Emacs \\
\hline
Mac OS & MacTeX & TeXShop \\
\hline
\end{tabular}

%控制栏位的宽度,将其对齐方式参数从l、c、r改为p{宽度}.
%实际这里如果没有竖线'|'那么表格中字体还是按照表的方式进行只是没有竖线分割。
\begin{tabular}{|p{100pt}|p{100pt}|p{100pt}|}
\hline
操作系统 & 发行版 & 编辑器 \\
\hline
Windows & MikTeX & TeXnicCenter \\
Unix/Linux & TeX Live & Emacs \\
Mac OS & MacTeX & TeXShop \\
\hline
\end{tabular}

%嵌套的表格
\begin{tabular}{p{100pt}p{100pt}p{100pt}p{100pt}}
\hline
操作系统 & 发行版 & 
%第一层子表格
\begin{tabular}{c}
常用工具 \\
	\hline
	\begin{tabular}{c|c}
	编辑器 & SHELL \\
	\end{tabular}
\end{tabular} \\ %不要忘记这里的\\

\hline
Windows & MikTeX & TeXnicCenter & Bash \\
Unix/Linux & TeX Live & Emacs Sh \\
Mac OS & MacTeX & TeXShop Tsh \\
\hline
\end{tabular}

\end{document}
