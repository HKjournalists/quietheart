%%%%%文档类(documentclass)有三种:article、report、book%%%%%
%\documentclass[options]{class} %文档类声明

%\usepackage[options]{package}  %引入宏包

%%%%%文档正文部分的开头通常有标题、作者、摘要等信息,之后是章%%%%
%%%%%节等层次结构,内容则散布于层次结构之间%%%%
%\begin{document}               %正文
%\title{标题}
%\author{作者}
%\today
%\maketitle

%\begin{abstract}
%...摘要部分...
%\end{abstract}

%%%%%常用的层次结构%%%%%
%\chapter{...}
%\section{...}
%\subsection*{...}%不让此章节显示在目录中加*
%\subsubsection{...}

%\end{document}


\documentclass[10pt,titlepage]{report} %文档类声明
%\usepackage{CJK(utf8)}
\usepackage{CJK(GBK)}
\begin{document}               %正文
%一些特殊字符需要转义,其中\textbackslash是反斜线'\',\\是换行或断行。
\# \$ \% \^{} \& \_ \{ \} \~{} \textbackslash \\
		hello!
		hello!%直接回车这样不会换行
		hello!\newline
		hello2!%用\newline,这样就断行了,这是硬断行

		hello22!%用两个空行,也断行了,但这实际是另起一段了,所以有缩进。
		hello22!\newpage
		hello,newpage!%用\newpage,这样开始了一个新页

		%用来设置断词BASIC 这个词不能断开,而 blar-blar-blar 可以在-处断开
		\hyphenation{BASIC blar-blar-blar}
		BASICBASICBASICBASICBASICBASICBASICBASICBASICBASICBASICBASICBASICBASICBASICBASICBASICBASICBASICBASICBASICBASICBASICBASICBASICBASICBASICBASICBASICBASIC%这样效果就是显示不了BASIC截断了。
		ttttttttttttttttttttttttttttttttttttttttttttttttttttttttttttttttttttttttttttttttttttttttttttttttttttttttttttttt%这个t也不显示
		\newline
		hi,你好

\end{document}
