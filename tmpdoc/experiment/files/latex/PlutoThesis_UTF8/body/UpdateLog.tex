% -*-coding: utf-8 -*-

\defaultfont

\BiChapter{模板升级、修改记录}{Update Record of the Thesis Model}
\label{Updatelog}

\BiSection{说明}{Introduction}
\label{Update:intro}
为了更加有效的维护该论文模板,特增加此章,用以记录模板所经历的改动,
同时此章也有助于用户更深入的了解该模板。

为了让更多的同学分享到最新的论文模板,建议大家在使用模板时如果对模板
有任何改动或者建议,都别忘了到紫丁香BBS上TeX版把自己发现或建议与大家
分享一下。

本章的记录包括版本升级、bug修复等任何涉及到模板内容的改动。

本模板起初是~Stanley~在~2005~年~\url{http://cvs.hit.edu.cn}~上创立了~Pluto(冥王星)哈尔滨工业大学博士学位论文模板开源项目。200~6年~\url{http://cvs.hit.edu.cn}~迁到了~\url{http://gf.cs.hit.edu.cn},nebula~也随之将该项目转移至此。2008年~2 月  ~\url{http://gf.cs.hit.edu.cn} 出现故障暂停服务,~luckyfox 将项目迁移到~ code.google~网站上,网址为 ~\url{http://code.google.com/p/plutothesis/}。若以后~\url{http://gf.cs.hit.edu.cn}~ 恢复服务,可能两个网站同时更新,不过建议大家在检查新版本时最好两个网站都查看一下。

\BiSection{存在的问题及新版本特色}{Problems to be Solved and the features of the latest version}

当前发布的版本修正了一些网友在使用前期版本时发现的bugs,并且功能更进一步增强,
使用更加方便,建议大家更新到该版本。当前版本目前没有已发现的问题存在。(详细内容请看后面的更新日志。)

其他未知待解决的问题还有赖于大家的使用和发现,共同完善。

\BiSection{版本历史}{The Version history about the template}
本节将对模板的版本号及升级记录及其更新者做一详细的说明。

\begin{hitlist}
\item UFO 模板为~1.0 版本。
\item cucme 模板为~1.1 版本。
\item nebula 模板为~1.2 版本。
\item Stanley 模板为~1.3 版本。
\item nebula 先后完善模板为~1.4、1.5 版本。
\item luckyfox 先后完善模板为~1.6、1.7rc1 版本。
\item jdg 完善版本为~1.7rc2 版本,luckyfox 整理发布。
\item luckyfox and LaTeX 发布~1.7 正式版本。
\item luckyfox and LaTeX 发布~1.8 rc1 版本。
\item luckyfox and LaTeX 发布~1.8 rc2 版本。
\item 做成真正的模板后为3.0版本,之后用``$\pi$''的值作为版本号,以后每升级一次精确度进一位,这是
借鉴\LaTeX{}的版本记录方法,象征着趋于完美。
\end{hitlist}

%%%%%%%%%%%%%%%%%%%%%%%%%%%%%%%%%%%%%%%%%%%%%%%%%%%%%%%%%%%%%%%%%%%%%%%%%%%%%
\BiSubsection{模板的诞生}{The Naissance of the Template}
本模板是网友UFO等(2004)基于清华大学博士论文模板,
按照哈尔滨工业大学论文规范开发的\LaTeX{}论文模板。

%%%%%%%%%%%%%%%%%%%%%%%%%%%%%%%%%%%%%%%%%%%%%%%%%%%%%%%%%%%%%%%%%%%%%%%%%%%%%
\BiSubsection{版本升级至$\gamma$~(by cucme--2005.06.06)}{Version Update $\gamma$~(by cucme--06.06.2005)}
\label{Update:06.06.05}

\BiSubsubsection{章节标号}{Mark of Chapter}
对于没有章标号的章,如结论等,定义了一个相应的命令\verb"\BiAppendixChapter"。

在这些命令中均含有两个参数,第一个为中文题目,第二个为英文题目。与UFO的最大不同在于,本模版直接生成中英文目录。

\BiSubsubsection{列表环境}{List Environment}
本模版将3个传统的列表环境参数作了修改,因此可以直接使用它们。不过有以下问题:
\begin{hitlist}
\item 缩进的具体参数可能有点误差,现在是按两个字$24pt$来缩进的,而实际上应该是两个字加上两个字间距。请朋友们试用后再修改吧。

  还有就是每个列表的item中的非首段没有缩进,我的临时解决办法是使用$2$个全角空格`` ''来模拟缩进。

\item 这是当前hitlist环境的第二个item,上一段就是使用$2$个全角空格`` ''来模拟缩进的。
\end{hitlist}

\BiSubsubsection{参考文献}{Reference}
模板中使用的是紫丁香网友Stanley提供的~chinesebst.bst。作了以下修改:
\begin{hitlist}
\item 修正了引用书籍不输出页码问题
\item 修正了引用博士、硕士论文不输出页码的问题
\item 修正了引用博士硕士论文的学校和学位类别颠倒的问题
\item 引用书籍版次位置不正确的问题
\item 使用缩写期刊名时吞掉"."问题
\end{hitlist}
还存在的问题:
\begin{hitlist}
\item 中文文献作者多于3个时输出的是et al 而不是"等",(我google了一下,貌似要用hooklee编的一个程序fixbbl来搞定,哪位试试吧。)
\end{hitlist}

目前可以这么临时解决修改bbl文件,最后版本的时候把中文出现et.al的地方用``等''代替。
保存一份main.bbl文件,以后用这个文件代替同名文件就可以了。

另外多于三个作者的英文文献没有发现输出不一致的问题,
可以再讨论一下。

%%%%%%%%%%%%%%%%%%%%%%%%%%%%%%%%%%%%%%%%%%%%%%%%%%%%%%%%%%%%%%%%%%%%%%%%%%%%%
\BiSubsection{版本升级至1.2(by nebula--2005.06.28)}{Version Update 1.2 (by nebula--28.06.2005)}
\label{Update:28.06.05}
这次升级主要是把近期关于该模板的一些修改整合进模板,同时增加了一些
介绍性的文字和例子。

\BiSubsubsection{模板内容的修改}{Update on the Content of the Model}
\begin{hitlist}
\item 重写了第一章软件环境介绍部分;
\item 第二章打印部分增加了关于Page Scaling选项的说明;
\item 第二章增加了一些公式的例子;
\item 增加了第三章“模板修改记录”,将校庆版的改动记录进来;
\item 增加了Unix/Linux下的clean方法,增加了一个Makefile文件,\$~make clean即可;
\end{hitlist}

\BiSubsubsection{模板格式的修改}{Update on the Format of the Model}
\begin{hitlist}
\item 在package.tex中把hyperref宏包的设置部分移到最后,避免与其它宏包
的冲突,解决了书签、目录链接不正确的问题;
\item 解决了书签的另一个问题,在点各个使用BiAppendixChapter的附录或
摘要时,标题总是被跳过去的,修改了Definition.tex和format.tex;
\item 解决了“定义”、“性质”等序号错乱的问题,修改了format.tex文件;
\item 去掉了关键字和Key Words后面的冒号;
\item 中文封页下面“研究生”等字按要求改为黑体;
\item 英文封页下边左侧的文字同样改为黑体字;
\item 增添了使用受权书的目录项和书签项;
\item 解决了目录细点、粗点问题,使用的是Stanley提供的方法1和2;
\item 增加了目录abstract后面的空行;
\item 调整目录行距;
\item 解决了CONTENTS和ABSTRACT大写的问题;
\item 调整了目录中点之间的距离使之更符合工大论文要求;
\end{hitlist}

%%%%%%%%%%%%%%%%%%%%%%%%%%%%%%%%%%%%%%%%%%%%%%%%%%%%%%%%%%%%%%%%%%%%%%%%%%%%%
\BiSubsection{版本升级至1.3~(by Stanley)}{Version Update 1.3 (by Stanley)}

在~\url{http://cvs.hit.edu.cn}~上创立了~Pluto(冥王星)项目, 以利于模板的发布和修改。

进行了下面这些修改:
\begin{hitlist}
\item 小小节的标题形式是和段落在一起的,并且不出现在目录中;
\item ``第1章''变成``第~1~章'',原来的在format.tex中已经修改,但是好像
   忘了将后面的删除了,也就是\verb"\chaptername"定义了两次,大家可以看看;
\item main.tex中的格式定义内容都放到了format.tex文件中;
\item 增加了yap使用开关,当为true时,使用yap查看时生成超级链接;
\item 在definition.tex中,增加了中文破折号命令\verb"\cdash",大家可以看看;
\item 页眉``第1章''和``章标题''之间增加了两个空格;
\item 封面的对齐方式等进行了微调;
\item 将format.tex definition.tex package.tex中的一些注释去掉了,
   由于经过多次更改,变得到处都是注释,使得内容比较乱,以后都将更改的
   内容写到ChangLog里面吧;
\item 增加了有章节的附录命令\verb"\BiAppChapter",使用方法参考appA.tex;
\item 增加了hitlist列表环境和publist列表环境;
\item 修改和完善了makefile文件;
\item 修改了各章节的使用说明等;
\item 增加了版权声明章节;
\item 首封增加了工大的logo,谁能贡献一个好点的logo?
\end{hitlist}
%%%%%%%%%%%%%%%%%%%%%%%%%%%%%%%%%%%%%%%%%%%%%%%%%%%%%%%%%%%%%%%%%%%%%%%%%%%%%
\BiSubsection{版本升级至1.4~(by nebula)}{Version Update 1.4 (by nebula)}
解决了linux+TeXlive环境下可能遇到书签乱码的问题,感谢理工大学的Huskier
网友发现该问题并提供了解决方案,感谢水木清华网友snoopyzhao提供的gbk2uni
程序代码。

模板的改动如下:
\begin{hitlist}
\item 增加了一个目录~tools,其中有三个文件,其中有两个是源文件,
gbk2uni是可执行文件,编译环境是gcc 3.2.2,如果运行有问
题请自行编译;
\item 改动了makefile文件;
\item 改动了本文件。
\end{hitlist}
%%%%%%%%%%%%%%%%%%%%%%%%%%%%%%%%%%%%%%%%%%%%%%%%%%%%%%%%%%%%%%%%%%%%%%%%%%%%%
\BiSubsection{版本升级至1.5~(by nebula)}{Version Update 1.5 (by nebula)}
更正了封面页中英文副导师、联合培养导师的格式问题,修正了中文副导师位置注释的
错误,感谢Huskier发现该bug,感谢TeX提出解决方案。

模板改动如下:
\begin{hitlist}
\item 改动了format.tex文件;
\item 改动了cover.tex文件;
\item 改动了本文件;
\item 为了方便shell的自动补齐操作,将makefile的文件名改为Makefile。
\end{hitlist}
%%%%%%%%%%%%%%%%%%%%%%%%%%%%%%%%%%%%%%%%%%%%%%%%%%%%%%%%%%%%%%%%%%%%%%%%%%%%%
\BiSubsection{版本升级至1.6~(by luckyfox)}{Version Update 1.6 (by
luckyfox)}
这里的更新大部分来自~jdg@lilac~的贡献,特别感谢他对本模板的关注。另外,对本次更新做出贡献的还有pineapple,TeX,lofe,luckyfox等。

模板改动如下:
\begin{hitlist}
 \item 增加了一个文件~make.bat~方便用户熟悉在~MSwindows~ 下编译模板的全过程,根据~main.tex~中~\verb|\def\useyap{true}|~还是\verb|\def\useyap{false}|自动选择生成书签的编译命令,减小入门困难,并为全局编译提供方便;
 \item 采用~jdg~修正过的~chinesebst.bst~文件,所有已发现的参考文献问题全部解决;
 \item 增加了~jdg~提出的中英文目录在书签中自动生成的功能;
 \item 增加了~jdg~的中英文图形标题索引的功能;
 \item 解决了TeX@lilac发现章节标题过长引起的目录问题;
 \item 增加了ToTemplateMaintainers.tex一章专门介绍pluto模板维护的一些问题,让用户了解模板维护的一般过程,吸引用户参与模板的维护更新;
 \item 增加了研究生院增加保密管理设置页,这里还有待研究生院论文规范的完善。具体说明见../body/authorization.text头部。
 \item 改动的文件有~main.tex、definition.tex、 package.tex、 format.tex、 Update-Log.tex、
  chinesebst.bst、Tricks.tex~等文件。
\end{hitlist}
%%%%%%%%%%%%%%%%%%%%%%%%%%%%%%%%%%%%%%%%%%%%%%%%%%%%%%%%%%%%%%%%%%%%%%%%%%%%%
\BiSubsection{版本升级至1.7rc1~(by luckyfox)}{Version Update 1.7rc1~(by
luckykfox)}

模板改动如下:
\begin{hitlist}
 \item 修正了长表格标题带来的中英文表格目录混乱的问题;
 \item 中文图表目录``插图'' 和``表格'' 加上空格,与``摘要''等协调;
 \item 调整~make.bat 中的命令,提前把上次生成的~dvi、ps 和~ pdf 文件删除,避免编译失败时误以为是本次的编译的问题;
 \item 修正~libq@lilac 发现的采用~Pineapple@lilac 的授权书与本模板不协调,带来的页眉为“博士期间发表的博士论文”的问题;
 \item 增加cmap宏包,可以制作中文可复制的~pdf 文档;
 \item 采用了标准的~ifpdf 宏包代替~ifpdf 定义;
 \item 增加arydshln宏包,给分块矩阵画虚线;
 \item 修正了libq 发现的授权书的硕博士论文相关的笔误问题;
 \item 把~gb\_452.cpx 和 ~gb\_452.cap 里面的中文图表索引每章后面的空行去掉了,与英文保持一致;
 \item 修正了make.bat中编译时得到的纸型是~letter 而不是~a4 的问题;
 \item 修正了长标题项中没有对齐的问题;
 \item 增加一个~ToDoList文件,方便模板维护者统计bug,决定下一步的工作动向;
 \item 修改宏包~hyperref 的生成书签选项,将~dvipdf 改成~dvips ,hyperref 作者反对使用前者。
 \item 修正~make.bat 中的~dvips 命令,去掉~-Pdf 选项(嵌入字体),这个严重影响生成~pdf 文件的速度,却没有太大必要。
\end{hitlist}
%%%%%%%%%%%%%%%%%%%%%%%%%%%%%%%%%%%%%%%%%%%%%%%%%%%%%%%%%%%%%%%%%%%%%%%%%%%%%
\BiSubsection{版本升级至1.7rc2~(by jdg)}{Version Update 1.7rc2(by jdg)}

模板改动如下:
\begin{hitlist}
    \item figures 目录: hit\_logo.pdf, hit\_logo.eps 替换成矢量的;
  \item chinesebst.bst 改正了两处,修正以前版面上提出的~url 问题;
  \item main.tex 加入~reference.bib for winedt gather 设置,在~winedt 可以使用~tree、gather 等特性;
    \verb|\graphicspath{{figures/}}|(定义所有的~eps 文件在 figures 子目录下)放到~\verb|\begin{document}| 之前,在使用~winedt 块编译的时候有用;第~1 章右开
  \item package.tex 增加 ~\verb|\usepackage{etex}|,增加计数器总数(原来是~256,宏包多,可能不够用),编译需基于~eTeX,因为咱模板计数器使用快超过~256了,如果用户自己在添加几个,编译就出错了。
  \item definition.tex 增加一个环境~formulasymb,用来对公式中的符号进行描述,原模板中与工大论文要求的有出入;重新定义~BiChapter 命令,实现标题手动换行,但不影响目录;调整子图编号,符合工大论文要求;
        增加一个命令~\verb|\dif|,在数学模式中输入微分~$\dif$;调整破折号~\verb|\cdash| 的长度;更新表格目录中长表格超链接失效的问题;
        微调表格标题上下的间距;
  \item format.tex 虽然无法像word一样用难看的黑体英文,但最少也要把 章标题 与小节标题的英文字体一致起来;调整中英文目录,现在1.7rc1中的中文目录,章标题 后产生空白,应该在章标题之前产生;
       增加一个命令~\verb|\citeup| 使显示的引用为上标形式,原来有一个~\verb|\ucite|,但~\verb|\ucite|在~winedt 默认设置里没有提示
            而~\verb|\citeup|就有,当然通过改~winedt,也可以使~\verb|\ucite|有;
  \item 完善~clean.bat, 重写了~make.bat文件,通过识别~main.tex中~\textbackslash usewhat的定义,自动选取合适的编译方式,
       支持~pdfLaTeX、dvips、dvipdfmx 三种编译方式及~yap方式。
  \item 更新本~updatelog.tex 文件;
  \item 新编译的~readme.pdf 替换原来的原来的~readme.pdf 字体嵌入不全,估计有的系统会有问题。
  \item 重新定义 ~\verb"\BiChapter" 命令,允许章标题过长时正文中手动换行,同时目录中自动换行。
\end{hitlist}
%%%%%%%%%%%%%%%%%%%%%%%%%%%%%%%%%%%%%%%%%%%%%%%%%%%%%%%%%%%%%%%%%%%%%%%%%%%%%
\BiSubsection{版本升级至 v1.7~(by luckyfox and LaTeX)}{Version Updatev1.7 (by luckyfox and LaTeX)}

模板改动如下:
\begin{hitlist}
  \item 支持二级图形目录,二级表格目录可以仿照图形目录实现;
  \item 子图形英文标题用法更改,由~\verb|\SubfigureCaption|变成~\verb|\SubfigEnCaption|,并修正此命令解决由此带来的鲁棒性可能不强的问题;
  \item 使用~violetwind@bbs.hit 提供的~linux 下的~makefile;
  \item 章节目录和书签中增加图表目录;
  \item 完善模板使用说明;
  \item 至此,目前已知的~bugs 都已解决,功能也日益完善。
\end{hitlist}
%%%%%%%%%%%%%%%%%%%%%%%%%%%%%%%%%%%%%%%%%%%%%%%%%%%%%%%%%%%%%%%%%%%%%%%%%%
\BiSubsection{版本升级至 v1.8rc1~(by luckyfox and LaTeX)}{Version Update v1.8rc1(by luckyfox and LaTeX)}

模板改动如下:
\begin{hitlist}
  \item 添加硕士学位论文支持,自此后硕博士论文模板合为一体,原硕士学位论文模板放弃维护;
  \item 添加研究生院官方学位论文规范(~doc 和~pdf 版本)到模板的附带文件中;
  \item 增加了针对模板的~WinEdt 的~gather,tree infterace 和自定义章节的关键词高亮正常显示的功能;
  \item 增加了~ xl2latex (从~ excel 表格到~ latex 表格代码的转换文件);
  \item 增加了一些入门的文档介绍及编辑技巧说明;
  \item 增加了对校内的模板现状的说明,提出一些选择模板的建议。
\end{hitlist}
%%%%%%%%%%%%%%%%%%%%%%%%%%%%%%%%%%%%%%%%%%%%%%%%%%%%%%%%%%%%%%%%%%%%%%%%%%

\BiSubsection{版本升级至 v1.8rc2~(by luckyfox and LaTeX)}{Version Update v1.8rc2(by luckyfox and LaTeX)}

模板修正以下bugs:
\begin{hitlist}
  \item 修正bst文件,使参考文献里书籍、学位论文中年份与页码之间是冒号,而不是逗号(参见规范),同时对学位论文进行细化,针对中英两种情况,
中文输出``大学名称论文级别'',而英文输出``论文级别,大学名称'';
  \item 修正图题,表题的字号问题。使用ccaption以来,图题,表题字号一直不正确,主要在format.tex进行修正;
  \item 把原 format.tex 的博硕一些定义,移到 type.tex;同时修正当学科不是engineering的时候,英文封面却始终显示 engineering的 小bug;
  \item 彻底修正附录的页眉问题;( 原因在于fancyhead设置是一个全局的设置,改变局部设置用\verb|\markboth{}{}|,在 Authorization.tex 添加了一项这个。在acknwledgement.tex去除fancyhead
  同时在format.tex页眉部分简化,definition.tex biappchapter去除markboth,没有必要);
  \item 修正硕士单面打印时,图表书签的链接指向问题,并去除单面打印时封面的空白页;
  \item 修正 format.tex 中定理的定义。 定理后面不用冒号;
  \item 修正图形英文标题的缩写, 由 ``Fig'' 改为``Fig.'';
  \item 调整 cmap 宏包的引用位置,适应 miktex 2.5。
\end{hitlist}

模板功能增强主要有:
\begin{hitlist}
  \item 增加CJKpunct宏包,使得中文标点符号的处理,更符合中文习惯;
  \item 取消原先的parlist宏包,采用enumitem(个人认为比parlist宏包强大,好用!),
         同时修正 itemize enumerate description 这些列表项的格式;
  \item 增加导言区使用中文的命令设置;
  \item 调整~main.tex 内容,图标索引分离到~figtab.tex,硕博士一些不同选项分离到~type.tex;
  \item 英文封面的 学科、单位,调整到cover.tex, 无需在format.tex中进行更改, 统一在cover.tex中进行更改,体现LaTeX的样式与内容分离的思想。
   需要注意的是:如果学科,单位中需要换行请用\verb|\newline|,而不是\verb|\\|,两边对齐(充满),用\verb|\hfill|.
  \item 针对WinEdt编辑器,修改swithes.dat,winedt.gdi文件,补充一些关键词,如:\verb|\citeup|,\verb|\ucite|,增强了gather功能;
  \item 增加了对 winedt5.5 中自定义命令在 tree 和 gather 中的 toc 的支持,使用方法见该目录下的文本说明。
\end{hitlist}

文档说明完善主要有:
\begin{hitlist}
  \item 补充论文规范里参考文献示例的条目到模板中,同时完善了正文里参考文献引用的使用说明;
  \item 对 Tricks.tex 中封面内容部分、参考文献部分进行了一些补充;
  \item 对校内TeX资源的连接的介绍做一些修正和补充;
\end{hitlist}

\BiSubsection{版本升级至~v1.8.0.20080228~(by luckyfox and LaTeX )}{Version
Update v1.8.0.20080228 (by luckyfox and LaTeX )}

模板修正以下bugs:
\begin{hitlist}
    \item 在使用dvipdfmx编译时,通过 \AtBeginDvi{\special{pdf:tounicode
        GBK-EUC-UCS2}} 可以不用gbk2uni.
        但使用hyperref宏包时,其unicode选项会使 GBK-EUC-UCS2失效,为此去掉
        unicode选项。 其他编译方式 仍需gbk2uni。
    \item 解决 muzak@lilacbbs 提出的 中英摘要关键词过长,
换行时不能自动缩进的问题。 为此在format.tex 对关键词加上悬挂缩进。
    \item 修正论文中url的网址与正文字体不同的bug,并给出一例;
    \item 修正多个附录时,英文目录存在的问题,都是Appendix A;
    \item 更正参考文献 书的版次问题 , reference.bib 中文书edition={第二版},英文书 edition={2nd}
    \item violetwind@hit  改进英文子图图题居中
    \item 修正 chinesebst.bst 文件 对英文硕士论文的处理,输出顺序与博士论文一致,先是学位级别,后是学校。
    \item 精调一下 中英图题间的行距 -1.3ex ;
    \item 重新设定公式与上下文的间距,原先是12pt,现改为10pt
    \item 解决由muzak提出的 当使用子图标题中包含公式符号时编译出错的问题。
原因:加入目录时 \verb|\xdef| 与\verb|\protect| 命令不兼容。 使用LaTeX中的 \verb|\protected@xdef|  代替原来的\verb|\xdef|。
具体参见TeX FAQS: \url{http://www.tex.ac.uk/cgi-bin/texfaq2html?label=edef}
	\item 硕士论文封面问题
	\item 章标题中的数学符号在正文和目录中加粗;节标题中的数学符号在正文中加粗,在目录中不加粗
	\item 中英目录中章标题后粗点还是细点?模板中提供了两种方案,现在采用细点方案,即中英目录中章标题后全部采用细点,中英一致!
	\item 增加两个表格字号切换命令,\verb+\normalbiao+~正常字号;\verb+\wuhaobiao+~五号字。 正文中默认使用\verb+\wuhaobiao+ 。
表格前后无需 \verb+\wuhao,\defaultfont+, 老用户替换 format.tex 即可。
  \item 修正了算法的标题编号问题,和样式问题。
\end{hitlist}


模板功能增强主要有:
\begin{hitlist}
  \item 将原 EditTools tools ThesisCriterion 归到 Accessories 目录里,规范附属文件;
  \item 删除原先algo.sty宏包,采用新算法宏包 algorithm2e,例子直接用的版面上worldguy提供的;
  \item colorlinks 由true全改成 false 吧, 毕竟用 false的人多一些;
  \item 参考文献标题自动大小写功能补充: 添加了三个虚词 via vs its,去掉不常用的 can;
  \item 增加 booktabs 宏包,用于做三线表格;
  \item 精简definition.tex 去除多余的 \verb|\makeatletter,\makeatother| ;
  \item 调整模板的版本标号形式,为*.*.*.*的形式,如1.8.0.20070910。编号规则是:其中第1位是大编号,如果研究生院对论文规范做大规模的调整,论文模板跟进,那么加1;第2位是小编号,如果有很多bugs修正,或者是功能结构上大的调整,则加1,第3位是一些小bug修正后,很快就发布的版本;后面的是模板发布的日期,方便网友使用,也方便管理员查看;
  \item 增加~relsize~宏包,方便调整个别公式字体大小;增加一个环境~flualign~,用于公式左对齐。
\end{hitlist}

文档说明完善主要有:
\begin{hitlist}
  \item 完善文档,增加 一小节,模板 FAQs;
  \item 参考文献 针对中英书籍版次,增加两个例子;
  \item 本次 bug 修正时,有些格式的调整在文档说明的正文中同时做了说明。
\end{hitlist}

%%%%%%%%%%%%%%%%%%%%%%%%%%%%%%%%%%%%%%%%%%%%%%%%%%%%%%%%%%%%%%%%%%%%%%%%%%
\BiSubsection{版本升级至~v1.8.1.20080528~(by luckyfox and LaTeX )}{Version
Update v1.8.1.20080528 (by luckyfox and LaTeX )}

模板修正以下bugs:
\begin{hitlist}
    \item 允许公式出现在页面顶部;
    \item 增大表格内行距;
\end{hitlist}


模板功能增强主要有:
\begin{hitlist}
    \item 增加主要符号表;
\end{hitlist}

文档说明完善主要有:
\begin{hitlist}
   \item 在说明文档中加上google的版本库地址;
   \item 增加红色的版本更新的提醒。
\end{hitlist}

%%%%%%%%%%%%%%%%%%%%%%%%%%%%%%%%%%%%%%%%%%%%%%%%%%%%%%%%%%%%%%%%%%%%%%%%%%
\BiSubsection{版本升级至~v1.8.2.20080601~(by luckyfox and LaTeX )}{Version
Update v1.8.2.20080601 (by luckyfox and LaTeX )}

此版本特别为庆祝六一儿童节,愿我们都保持一颗童心,开心快乐,真诚永远!

模板修正以下bugs:
\begin{hitlist}
    \item 目录页码加上小横线,和正文格式一致;
\end{hitlist}


模板功能增强主要有:
\begin{hitlist}
    \item 保持文件 GBK 编码的同时,增加对 xetex 的支持。
\end{hitlist}

文档说明完善主要有:
\begin{hitlist}
   \item 补充对 xetex 的一些简单说明;
   \item 修改推荐的软件信息,删除对 chinatex 的推荐,增加 MiCTeX 和 MiKTeX 软件。
\end{hitlist}
%%%%%%%%%%%%%%%%%%%%%%%%%%%%%%%%%%%%%%%%%%%%%%%%%%%%%%%%%%%%%%%%%%%%%%%%%%%
\BiSubsection{版本升级至~v1.8.3.20081210~(by luckyfox and LaTeX )}{Version
Update v1.8.3.20081210 (by luckyfox and LaTeX )}

模板修正以下bugs:
\begin{hitlist}
    \item 文件头部加入编码信息;
    \item 跨页表格,标题宽度,末行居中;
    \item 消除Introduction.tex中 \textbackslash section的\} 前没有空格导致出现的hyperref宏包相关错误; 
    \item jdg发现并修正五号表表格未能真正居中对齐的bugs;
    \item 修正UTF编码下某些注释的文字不完整;
    \item 修正子图标题中的字号非5号字的问题; 
    \item 修正utf8的main.tex中原由gbk转换时部分文字无法显示的小bug;
    \item [GBK+UTF8]将jdg针对多个子图并列时子图中英文caption可能不对齐的bug修正的子图英文caption定义代码加入;
    \item 增加gb\_452\_UTF8文件;
\end{hitlist}


模板功能增强主要有:
\begin{hitlist}
    \item 默认编译改成dvipspdf;
    \item 上传xelatex版本的readme,字体比较均匀;
    \item 创建PlutoThesis的utf8文件;
    \item 上传参考文献国标2005文件;
\end{hitlist}

文档说明完善主要有:
\begin{hitlist}
   \item  updatelog的更新
\end{hitlist}
%%%%%%%%%%%%%%%%%%%%%%%%%%%%%%%%%%%%%%%%%%%%%%%%%%%%%%%%%%%%%%%%%%%%%%%%%%
\BiSubsection{版本升级至~v1.9.0.20081213~(by luckyfox and LaTeX )}{Version
Update v1.9.0.20081213 (by luckyfox and LaTeX )}

模板修正以下bugs:
\begin{hitlist}
    \item 修正makefile,添加xelatex编译代码; 
    \item 去掉GBK封面上的"专业",英文封面的 in \textbackslash exueke; 
    \item 修改了版芯,部分尺寸调整还待验证; 
    \item 将英文目录的章改为四号字; 
    \item 第四级标题题序顶格写,与标题空一格,阐述内容另起一段; 
    \item 修正xelatex编译时,英文粗体采用了粗体,而非Times New Roman的bug;
    \item UTF8,修正xelatex编译中文乱码的bug,去掉hyperref的unicode选项; 
\end{hitlist}


模板功能增强主要有:
\begin{hitlist}
    \item 提交比较符合GBT7714-2005标准的bst文件,文件名加后缀"\_HIT"; 
    \item 中文封面加上学校代码和密级; 
    \item 加入新论文规范的readme.pdf;
    \item 添加jdg制作的chinesebst2005,包括GBK和UTF8编码;
    \item 根据参考文献标准要求,修改了Reference.bib文件完善示例文献条目;
\end{hitlist}

文档说明完善主要有:
\begin{hitlist}
   \item  updatelog的更新
\end{hitlist}
%%%%%%%%%%%%%%%%%%%%%%%%%%%%%%%%%%%%%%%%%%%%%%%%%%%%%%%%%%%%%%%%%%%%
\BiSubsection{版本升级至~v1.9.1.20090323~(by luckyfox  and maldinilz)}{Version
Update v1.9.1.20090323 (by luckyfox  and maldinilz)}

模板修正以下bugs:
\begin{hitlist}
\item 修正保密项和学校代码导致部分编译分时下两行间距拉大的问题;
\item 修正英文目录中手动加入的条目字号bug;
\item 修正示例中上下放置的子图题和总图题间距过大的问题;
\item 修正章节间距、段落间距、公式间距;
\item 将正文中子图的引用从图7-1 (a)修正为图7-1 a)
\end{hitlist}

模板功能增强主要有:
\begin{hitlist}
\item 调整封面中文行间距;
\item 增加一个将子图图题写到总图题之下的例子;
\item 精调正文行距;
\item 默认改为pdflatex编译;
\item 添加算法的双图题例子,添加算法双图题的定义;
\item 添加win下GBK版本可以用的gbk2uni.exe;
\end{hitlist}

文档说明完善主要有:
\begin{hitlist}
  \item updatelog的更新
\end{hitlist}

%%%%%%%%%%%%%%%%%%%%%%%%%%%%%%%%%%%%%%%%%%%%%%%%%%%%%%%%%%%%%%%%%%%%
\BiSubsection{版本升级至~v1.9.2.20090324~(by luckyfox and maldinilz)}{Version
Update v1.9.2.20090324 (by luckyfox and maldinilz)}

模板修正以下bugs:
\begin{hitlist}
\item 英文目录的章标题用小四号字加粗,不用四号字,规范上写错了; 
\item 修正结论和致谢的英文未用复数(Conclusions,Acknowledgements)的问题;
\item 致谢放在个人简历前面(按研院老师说明,而非论文规范中致谢在承诺之前);
\end{hitlist}


文档说明完善主要有:
\begin{hitlist}
  \item updatelog的更新
\end{hitlist}

%%%%%%%%%%%%%%%%%%%%%%%%%%%%%%%%%%%%%%%%%%%%%%%%%%%%%%%%%%%%%%%%%%%%%
\BiSubsection{版本升级至~v1.9.2.20090424~(by luckyfox)}{Version Update v.19.2.20090424 (by luckyfox)}

模板修正以下bugs:
\begin{hitlist}
\item 修正在format.tex中手动给出中文封面上的分类号的bug (调试的时候把相应命令改成确定数值了);
\item 更正UpdateLog中的一个conclusions拼写错误(不影响使用);
\end{hitlist}

模板功能增强主要有:
\begin{hitlist}
\item 完善模板对CJKpunct package的支持
\end{hitlist}
文档说明完善主要有:
\begin{hitlist}
\item updatelog的更新;
\item 在definition中对xelatex字体选择做些说明,将来最好在tricks.tex中再说明一下,给用户在不同平台下用xelatex编译提供帮助;
\item 对GBK版本的main.tex中第一句标示文件编码的语句进行补充说明,因为:“当main.tex文件中存在命令code:gb2312时,用winedt54打开该文件时会winedt中所有的编译按钮都是灰色的,tex文档中的注释、latex命令等的颜色高亮显示也不正常,感觉是winedt没有识别出tex文件一样,当把命令code:gb2312删除时,一切都正常了。”用winedt5.5不存在这个问题,应该说这是winedt的bug,但是为了照顾ctex用户,编码下增加这个说明;
\end{hitlist}
